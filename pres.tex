%% LyX 2.0.6 created this file.  For more info, see http://www.lyx.org/.
%% Do not edit unless you really know what you are doing.
\documentclass{beamer}
\usepackage[T1]{fontenc}
\usepackage[latin9]{inputenc}
\usepackage{amsmath}
\usepackage{amssymb}

\makeatletter

%%%%%%%%%%%%%%%%%%%%%%%%%%%%%% LyX specific LaTeX commands.
\pdfpageheight\paperheight
\pdfpagewidth\paperwidth


%%%%%%%%%%%%%%%%%%%%%%%%%%%%%% Textclass specific LaTeX commands.
 % this default might be overridden by plain title style
 \newcommand\makebeamertitle{\frame{\maketitle}}%
 \AtBeginDocument{
   \let\origtableofcontents=\tableofcontents
   \def\tableofcontents{\@ifnextchar[{\origtableofcontents}{\gobbletableofcontents}}
   \def\gobbletableofcontents#1{\origtableofcontents}
 }
 \def\lyxframeend{} % In case there is a superfluous frame end
 \long\def\lyxframe#1{\@lyxframe#1\@lyxframestop}%
 \def\@lyxframe{\@ifnextchar<{\@@lyxframe}{\@@lyxframe<*>}}%
 \def\@@lyxframe<#1>{\@ifnextchar[{\@@@lyxframe<#1>}{\@@@lyxframe<#1>[]}}
 \def\@@@lyxframe<#1>[{\@ifnextchar<{\@@@@@lyxframe<#1>[}{\@@@@lyxframe<#1>[<*>][}}
 \def\@@@@@lyxframe<#1>[#2]{\@ifnextchar[{\@@@@lyxframe<#1>[#2]}{\@@@@lyxframe<#1>[#2][]}}
 \long\def\@@@@lyxframe<#1>[#2][#3]#4\@lyxframestop#5\lyxframeend{%
   \frame<#1>[#2][#3]{\frametitle{#4}#5}}

%%%%%%%%%%%%%%%%%%%%%%%%%%%%%% User specified LaTeX commands.
\usepackage{listings}
\input{/accounts/gen/vis/paciorek/latex/paciorekMacros}
\usetheme{Warsaw}

\setbeamercovered{transparent}
% or whatever (possibly just delete it)

%\usecolortheme{seahorse}
%\usecolortheme{rose}

% seems to fix typewriter font in outline header:
\usepackage{ae,aecompl}
% or whatever (possibly just delete it)

%\usecolortheme{seahorse}
%\usecolortheme{rose}

% seems to fix typewriter font in outline header:

\makeatother

\begin{document}


\title[Statistical modeling of biomass increment~~~~\insertpagenumber]{Statistical modeling of biomass increment}


\author[Chris Paciorek]{Chris Paciorek\\
Department of Statistics; University of California, Berkeley\\
}


\date{June 2015}

\makebeamertitle

\lyxframeend{}


\lyxframeend{}\lyxframe{Model overview}
\begin{itemize}
\item Modeling Lyford 13 plots plus HF census data for 1960-2013
\item Two data sources and therefore two likelihood terms: census DBH and
ring increment
\item Trees not in census in a year are assumed < 5 cm dbh
\item DBH in a year (end of season) is equal to DBH in previous year plus
increment that year
\item Unknown true increment assumed to come from a distribution

\begin{itemize}
\item Overall mean increment for all trees varies by year (year effect)
\item Each tree also has a mean increment common to all years (tree effect)
\item Can have increment vary by taxon and with tree size (random taxon
effects + regression on size)
\item Playing with having increment be autocorrelated over time via AR process
\end{itemize}
\item Account for when in a season DBH is measured by scaling relative to
inferred seasonal growth curve
\item Estimated increment is sum across all trees

\begin{itemize}
\item Assume no missing trees given census
\end{itemize}
\end{itemize}

\lyxframeend{}


\lyxframeend{}\lyxframe{Model details}
\begin{itemize}
\item measured log DBH assumed to follow a $t$ distribution with 3 degrees
of freedom, centered on true log DBH
\item log increment assumed to follow a normal distribution centered on
true log increment, $\log X_{it}$
\item $\log X_{it}\sim t_{\nu}(\alpha_{i}+\gamma_{t}+\beta(D_{i,t-1}-30)I(D_{i,t-1}>30),\sigma_{i}^{2})$
(increment increases with size only for size > 30 cm)
\item $\alpha_{i}\sim N(\alpha_{\mbox{taxon}(i)},\sigma_{\mbox{taxa}}^{2})$
\item $\gamma_{t}\sim N(0,\sigma_{t}^{2})$
\item Not indicated above, but also can include residual increment process
as AR(1) process with AR parameter $\rho$, estimated in model.
\end{itemize}

\lyxframeend{}\lyxframe{Some results}
\begin{itemize}
\item Assessment of model fit and raw data has revealed some anomalies in
both increment and DBH data and inconsistencies in HF and Neil/Dan's
data; some of this resolved based on consultation with Neil/Dan
\item Initial fits to all individual trees in Neil/Dan plots: see tree\_plots.pdf
\end{itemize}

\lyxframeend{}


\lyxframeend{}\lyxframe{Still to be explored}
\begin{itemize}
\item Can we better pin down distinction between DBH error and increment
error
\item Increment error may be strongly autocorrelated (because of use of
two cores only)
\item Not clear how to incorporate long-term trends in increment (or if
needed)
\item Do increments decline as tree nears death?
\item Need to deal with uncertain year of death
\item Need to get allometry information
\item Would be good to assess impact of not having census data - loss of
some larger trees that are decayed and miss many small trees
\item How get overall stand estimate from having three plots? 
\item How incorporate info from nested rings?\end{itemize}


\end{document}
